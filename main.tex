\documentclass[12pt, oneside]{article} % for thesis
% \documentclass[twocolumn,10pt]{article}
\usepackage[utf8]{inputenc}
\usepackage{natbib}
\usepackage{graphicx}
\usepackage{stfloats} % for positioning of figure* on the same page
\usepackage{caption}
\usepackage{tikz}
\usepackage[inline]{enumitem}
\usepackage{amsmath}
\usepackage{subcaption}
\usepackage[breaklinks=true,colorlinks=true, allcolors=blue]{hyperref}
\usepackage{breakcites}
\usepackage{microtype}
\usepackage{lipsum}
\usepackage{xcolor}
\usepackage{array}
\usepackage{float}
\usepackage{adjustbox}
\usepackage{listings}
\usepackage{csquotes}
\usepackage{makecell}
\usepackage{pdfpages}
\usepackage{xcolor}
\usepackage[utf8]{inputenc}       % For UTF-8 encoding
\usepackage{listings}
\usepackage{amsmath}              % Optional: for math symbols
\usepackage{anyfontsize}
\usepackage{graphicx} % required for \scalebox

\lstdefinelanguage{Faust}{
    morekeywords={import, process, environment, declare, with, if, else, while, for, int, float, true, false},
    sensitive=true,
    morecomment=[l]{//}, % Line comment
    morecomment=[s]{/*}{*/}, % Block comment
    morestring=[b]", % Strings
}

% Customize the appearance of the code
\lstset{
    language=Faust,
    backgroundcolor=\color{lightgray!20},
    % basicstyle=\ttfamily\small,
    basicstyle=\fontsize{8pt}{9pt}\selectfont\ttfamily,
    keywordstyle=\color{blue}\bfseries,
    stringstyle=\color{orange},
    commentstyle=\color{green}\itshape,
    showstringspaces=false,
    % numbers=left,
    % numberstyle=\tiny,
    frame=single,
    breaklines=true,
      % basicstyle=\ttfamily,
  % literate={\delta}{{\(\delta\)}},
}


\captionsetup[lstlisting]{justification=centering, singlelinecheck=false}
\providecommand{\gls}[1]{#1}
\definecolor{highlight}{RGB}{00,150,00}
\definecolor{todo}{RGB}{200,50,00}
\newcommand{\SIMSESpec}{\texttt{SIMSE\_Spec}}
\newcommand{\LoneSpec}{\texttt{L1\_Spec}}
\newcommand{\JTFS}{\texttt{JTFS}}
\newcommand{\DTWEnv}{\texttt{DTW\_Envelope}}
\newcommand{\OutDomain}{\textbf{Out-Domain Generation}}
\newcommand{\BPNoise}{\textbf{BP-Noise}}  
\newcommand{\AddSineSaw}{\textbf{Add-SineSaw}}  
\newcommand{\AmpMod}{\textbf{Noise-AM}}  
\newcommand{\FMMod}{\textbf{SineSaw-AM}}  

\title{Out of Domain Experiments in Sound-Matching}
% \author{asalimi }
% \date{May 2025}

\begin{document}

\maketitle


\section{Experiment Setup}
\label{sec:experiment_setup}
% loss functions, target vs imitator, training loop
300 sound-matching experiments are conducted and the MSS values are recorded after 200 iterations of the sound-matching loop. From the 300 targets and outputs, we select 40 examples for manual ranking. 


\section{Out-of-Domain Examples}
We look at some examples of out-of-domain sound-matching experiments, discuss the evaluation methods, and analyze the loss function landscapes. 


\subsection{AM-Synthesizer matching with Non-Overlapping Frequencies}
For this experiment, we choose two instances of \FMMod, with non-overlapping frequency ranges of 1-250 Hz for the target synth and 1000-5000 Hz for the imitator. Following the experiment setup in Section~\ref{sec:experiment_setup}, we show the manual and automatically assigned ranks.

\subsubsection{A Closer Look at the Loss Function Landscapes}


\begin{lstlisting}[caption={\FMMod. "{Low Freq}" and "{High Freq}" are replaced with the lowest and highest allowed frequencies for the synth.}, label={lst:program3},language=Faust,float,floatplacement=!H,xleftmargin=1em,xrightmargin=0.5em,firstnumber=0,aboveskip=0em, belowskip=-1em]
import("stdfaust.lib");
carrier = hslider("carrier",100,"{Low Freq}","{High Freq}",1);
amp = hslider("amp",6,1,20,1);
sineOsc(f) = +(f/ma.SR) ~ ma.frac:*(2*ma.PI) : sin;
sawOsc(f) = +(f/ma.SR) ~ ma.frac;
process = sineOsc(amp)*sawOsc(carrier);
\end{lstlisting}

\bibliographystyle{alpha}
\bibliography{references}

\end{document}
